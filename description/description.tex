\documentclass[a4paper]{scrartcl}
\usepackage[utf8x]{inputenc}
\usepackage[T1]{fontenc}
\usepackage{url}
\usepackage[affil-it]{authblk}

\newcommand*{\email}[1]{\texttt{#1}}

\title{TTC'16 Live Contest Case Study}
\subtitle{Execution of dataflow-based model transformations}

\author{Antonio Garcia-Dominguez}
\affil{\small Aston University, Birmingham, UK \\ \texttt{a.garcia-dominguez@aston.ac.uk}}
\date{}

\begin{document}

\maketitle

\begin{abstract}
  Abstract.
\end{abstract}

\section{Introduction}
\label{sec:intro}

Model-to-model transformation frameworks and tools have evolved in the
recent years to process larger models more efficiently. Traditional
\emph{batch} transformation engines took the entire source model(s)
and produced the entire target model(s) exactly once. A small change
would require going again through the entire model. Alternatively, an
\emph{incremental} transformation engine processes only the change
itself, updating the target model(s) as needed.

Various approaches for incremental transformations are present in the
literature: for instance, ReactiveATL~\cite{tisi_lazy_2011} is an
alternative execution engine of ATL which only computes and updates
results on-demand as the model changes. VIATRA3 is a general purpose
framework for reactive transformations~\cite{bergmann_viatra_2015}, in
which users write rules that trigger as the model changes. Beyond
model transformations, IncQuery~\cite{bergmann_incremental_2010} is an
incremental query language that builds a RETE network and feeds EMF
change notifications to keep query matches up to date. Streaming
transformations have been studied by Cuadrado and
Lara~\cite{cuadrado_streaming_2013} with the Eclectic tool and its
ATL-like language, and by David, Rath, and
Varró~\cite{david_streaming_2014} through complex event processing.

Most of these systems can be seen as complex transformations of
rule-based systems into event-driven systems, or frameworks that allow
for writing these event-driven systems. This case suggests studying a
type of notation that may be more directly amenable to incremental
execution: a graph of model-oriented primitives which generate and
process streams of tuples. The notation is inspired on popular
Extract-Transform-Load (ETL) tools such as Pentaho Data
Integration\footnote{\url{http://community.pentaho.com/projects/data-integration/}}
or Talend Data
Integration\footnote{\url{https://www.talend.com/products/data-integration}}. As
data integration can be considered a form of model transformation,
perhaps there might be lessons to learn from these languages. In
addition to their potentially simpler incrementality, breaking rules
into smaller primitives might increase the level of detail of the
execution traces of the transformation.

This case study presents a simplistic version of such a notation, with
primitives subdivided into ``minimal'' and ``extended''
sets. Participants are tasked with writing an execution engine using
their favorite approach (e.g. code generation or model interpretation)
and tool, which may support batch and/or incremental transformations.

All the resources are included in the official Github
repository\footnote{\url{https://github.com/TransformationToolContest/ttc16-live-contest}}. These
resources are mentioned in Section~\ref{sec:case}. The evaluation
criteria for the provided solutions are described in
Section~\ref{sec:eval}. Contestants are invited to raise questions
through GitHub issues should there be any unclear points in the
description.

\section{Case description}
\label{sec:case}

\subsection{Abstract syntax}
\label{sec:asyn}

Explain the abstract syntax of the language in rough terms, and also
the primitives. Should mention the expected properties of a model
element (eContainer and eClass so far).

\subsection{Concrete syntax}
\label{sec:csyn}

Explain the textual (Xtext-based) and graphical (Sirius-based)
notations. Doesn't need to be too detailed, as contestants won't
actually write transformations.

\subsection{Example: Families to Persons}
\label{sec:f2p}

Explain the Families2Persons example as described through the
notations.

\subsection{Task 1: base primitives}
\label{sec:baseprim}

% tree2graph: AllInstances, Evaluate, Filter, NewInstance, SetFeature
% families2persons: AllInstances, Evaluate, Filter, NewInstance, SetFeature

families2persons could be implemented, then tree2graph could be the
``extra'' transformation for soundness.

\subsection{Task 2: extended primitives}
\label{sec:extprim}

% flowchart2html: base + ForEach, AddToContainer
% flowchart2html-alt: flowchart2html + conditional expressions
% class2rdb: base + AddToContainer

flowchart2html could be implemented, then class2rdb could be the
``shadow'' transformation for soundness.

\section{Evaluation}
\label{sec:eval}

Correctness (w/shadow transformation with base primitives revealed
during conference), performance, extensibility.

\bibliographystyle{plain}
\bibliography{bibliography}

\end{document}
